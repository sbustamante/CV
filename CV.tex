\documentclass[10pt,a4paper]{moderncv}
\moderncvtheme[blue]{classic}      
\usepackage[scale=0.8]{geometry}
\usepackage{color}
\definecolor{myblue}{RGB}{55,115,179}
\recomputelengths

%Personal Data
\firstname{\Large{\textbf{\textit{Sebasti�n}}\vspace*{0.3cm} \\} }
\familyname{\huge{\textbf{Bustamante Jaramillo}}\\  \color{myblue}\rule{7.5cm}{0.5mm}}
\title{\large{Curriculum Vitae}}
\address{13355 Berlin, Germany}{Brunnenstrasse 129/130}
\mobile{+49 151 10403846}
\email{macsebas33@gmail.com}
\photo[65pt]{MyPicture.jpg} 
%\quote{\small I am a physicist with a strong interest in astrophysics, specifically in cosmology.
%I have always thought humanity's understanding the large-scale universe through science is like a
%bacteria colony trying to build a complete map of the earth or even more. I am looking forward to 
%cooperate and get involved into high quality research teams in order to contribute my humble grain 
%of sand to this grand and exciting enterprise called science.}
\nopagenumbers{}


%**************************************************************************************************

\begin{document}

\vspace{1.0cm}
\maketitle
%==================================================================================================

\section{\textsc{General Information}}

\cvitem{\small\textbf{Name}}	
	{ Sebasti�n Bustamante Jaramillo.}
\cvitem{\small\textbf{Date of Birth}}	
	{ 20$^{th}$ June, 1990.}
\cvitem{\small\textbf{Place of Birth}}	
	{ Maceo - Colombia.}
\cvitem{\small\textbf{Nationality}}		
	{ Colombian.}
\cvitem{\small\textbf{Marital Status}}	
	{ Single.}
\cvitem{\small\textbf{Identification}}	
	{ c.c. 1128400433}
% \cvitem{\small\textbf{Passport}}	
% 	{ AP117736}
\cvitem{\small\textbf{Address}}			
	{Brunnenstrasse 129/130, 13355 Berlin}
\cvitem{\small\textbf{Mobile}}			
	{+49 151 10403846}
\cvitem{\small\textbf{e-mail}}			
	{macsebas33@gmail.com}
\cvitem{\small\textbf{Inst. e-mail}}		
	{sebastian.bustamante@udea.edu.co}
   

\vspace{0.5cm}
%==================================================================================================

\section{\textsc{Education}}

\cvitem{\small\textbf{2006}}							
	{\textbf{High School Diploma}, \textit{Instituci�n Educativa Cisneros}, Cisneros, Colombia.} 	

\vspace{0.2cm}

\cvitem{\small\textbf{2007-2013}}							
	{\textbf{B.Sc. in Physics}, \textit{Physics Institute, Universidad de Antioquia}, Medell�n, 
	Colombia.}
\cvitem{\small\textbf{Thesis}}
	{``The place of the Milky Way and Andromeda in the cosmic web''.}
\cvitem{\small\textbf{Description}}
	{This study is aimed to characterize the local environment of Local Group (LG)-like systems from 
	dark matter simulations of the large-scale universe. Using two different types of simulations, an 
	unconstrained simulation (Bolshoi project) and a set of constrained simulations (CLUES project), 
	it is first constructed a LG-like sample based upon observational constrains on the kinematic 
	properties and isolation criteria of the real LG, along with the results of the CLUES simulations. 
	By using a tensorial scheme based upon the peculiar velocity field of the dark matter, the V-web 
	scheme, it is classified the local environment of systems in each simulation. Finally, it has 
	been found that LG-like systems lies preferentially in sheet-like regions, furthermore a 
	significant	environmental bias for the total mass and the specific energy. No correlations have 
	been found for the specific angular momentum and other studied properties.}
\cvitem{\small\textbf{Supervisor}}
	{Prof. Dr. Jaime E. Forero-Romero}

\vspace{0.2cm}

\cvitem{\small\textbf{2015-}}
	{\textbf{PhD in Astrophysics}, \textit{Heidelberg University}, Heidelberg, 
	Germany.}
\cvitem{\small\textbf{Supervisor}}
	{Prof. Dr. Volker Springel}

\vspace{0.2cm}
%--------------------------------------------------------------------------------------------------
\subsection{\textsc{Additional Education}}
\cvitem{\small\textbf{2008}}							
	{\textbf{Extension Course in Planetary Sciences}, \textit{Faculty of Exact and Natural Sciences, 
	Universidad de Antioquia}, Medell�n, Colombia.}


\vspace{0.5cm}
%==================================================================================================

\section{\textsc{Fields of Interest}}

\cvitem{}
	{Cosmology. Large-scale structure formation. Galaxy astrophysics. Planetary interior. Numerical 
	simulations. Computational astrophysics. General astrophysics and physics. Programming.}


\vspace{0.5cm}
%==================================================================================================

\section{\textsc{Languages}}

\cvitem{\small\textbf{Spanish}}
	{Native speaker.}
\cvitem{\small\textbf{English}}
	{Fluent.}
\cvitem{\small\textbf{German}}
	{Basic.}

\vspace{0.5cm}
%==================================================================================================

\section{\textsc{Computer Skills}}

\cvitem{\small\textbf{Systems}}
	{Linux, MSWindows.}
\cvitem{\small\textbf{Development}}
	{C, Python, Bash.}
\cvitem{\small\textbf{Software}}
	{Mathematica, LaTeX, gnuplot, Gadget.}
\cvitem{\small\textbf{Tools}}
	{N-body simulations, SPH, MonteCarlo, Finite Differences, Numerical integrators, Audio 
	processing.}
\cvitem{\small\textbf{Repositories}}
	{A list of my projects can be found in my \textit{github} page: 
	\url{https://github.com/sbustamante}.}


\vspace{0.5cm}
%==================================================================================================

\section{\textsc{Teaching}}

\cvitem{\small\textbf{TA}}
	{\textbf{Physics 1 (Newtonian Mechanics), 2013/I}, \textit{Faculty of Exact and Natural Sciences, 
	Universidad de Antioquia}, Medell�n, Colombia.}	
\cvitem{\small\textbf{TA}}
	{\textbf{Computational Complement of Physics 2 (Electricity and Magnetism), 2013/I}, 
	\textit{Faculty of Exact and Natural Sciences, Universidad de Antioquia}, Medell�n, Colombia.}
\cvitem{\small\textbf{TA}}
	{\textbf{Computational Complement of Physics 3 (Oscillations and Waves), 2013/I}, 
	\textit{Faculty of Exact and Natural Sciences, Universidad de Antioquia}, Medell�n, Colombia.}
\cvitem{\small\textbf{ Lect.}}
	{\textbf{Introductory Physics, 2013/II}, \textit{Faculty of Engineering, Universidad de 
	Antioquia}, Medell�n, Colombia.}
\cvitem{\small\textbf{ Lect.}}
	{\textbf{Laboratory of Physics 1 (Newtonian Mechanics), 2013/II}, \textit{Faculty of Pharmaceutics 
	Chemistry, Universidad de Antioquia}, Medell�n, Colombia.}
\cvitem{\small\textbf{ Lect.}}
	{\textbf{Introductory Physics, 2014/I}, \textit{Faculty of Engineering, Universidad de 
	Antioquia}, Medell�n, Colombia.}
\cvitem{\small\textbf{ Lect.}}
	{\textbf{Laboratory of Physics 1 (Newtonian Mechanics), 2014/I}, \textit{Faculty of Pharmaceutics 
	Chemistry, Universidad de Antioquia}, Medell�n, Colombia.}
\cvitem{\small\textbf{ Lect.}}
	{\textbf{Introductory Computation, 2014/II}, \textit{Faculty of Exact and Natural Sciences, 
	Universidad de Antioquia}, Medell�n, Colombia.}
\cvitem{\small\textbf{ Lect.}}
	{\textbf{Computational Methods for Astronomy and Physics, 2014/II}, \textit{Faculty of Exact and Natural 
	Sciences, Universidad de Antioquia}, Medell�n, Colombia.}
\cvitem{\small\textbf{ Lect.}}
	{\textbf{Computational Methods for Astronomy and Physics, 2015/I}, \textit{Faculty of Exact and Natural 
	Sciences, Universidad de Antioquia}, Medell�n, Colombia.}
	
\vspace{0.5cm}
%==================================================================================================

\section{\textsc{Honours, Awards, and Accomplishments}}

\cvitem{\small\textbf{2012}}
	{\textbf{First Best Oral Presentation}, \textit{II International Congress of Astrobiology}, 
	Medell�n, Colombia.}
\cvitem{\small\textbf{2013}}
	{\textbf{Best physics graduate student}, \textit{Universidad de Antioquia}, Medell�n, Colombia.}
\cvitem{\small\textbf{2015}}
	{\textbf{DAAD PhD Scholarship}, Germany.}

\vspace{0.5cm}

%==================================================================================================

\section{\textsc{Research Experience}}
 
\cvitem{\small\textbf{2010-2011}}	
	{\textbf{Young Investigator Programme}, Fundamental of quantum mechanics, \textit{Group of Atomic 
	and Molecular Physics (GFAM), Committee for Research Development (CODI), Universidad de Antioquia}, 
	Medell�n, Colombia.}
\cvitem{\small\textbf{2011-2012}}	
	{\textbf{Young Investigator Programme}, Thermal evolution of rocky exoplanets, \textit{Group of
	Computational Physics and Astrophysics (FACom), Committee for Research Development (CODI), 
	Universidad de Antioquia}, Medell�n, Colombia.}
\cvitem{\small\textbf{August, 2012}}	
	{\textbf{Research Internship}, \textit{Physics Department, Universidad de los Andes}, Bogot�, 
	Colombia.}
 
 
\vspace{0.5cm}
%==================================================================================================

\section{\textsc{Papers}}

\cvitem{\small\textbf{2015}}
	{\textit{Tensor anisotropy as a tracer of cosmic voids}, 
	\textbf{S. Bustamante}, J.E. Forero-Romero, MNRAS 453 
	1 497, 2015. \textbf{ADS:} \link{adsabs.harvard.edu/abs/2015MNRAS.453..497B}}
	\vspace{0.3cm}
\cvitem{\small\textbf{2013}}
	{\textit{The kinematics of the Local Group in a cosmological context}, J.E. Forero-Romero, Y. 
	Hoffman, \textbf{S. Bustamante}, S. Gottloeber, G. Yepes,
	ApJL 767 L5, 2013. \textbf{ADS:} \link{adsabs.harvard.edu/abs/2013ApJ...767L...5F}}
	\vspace{0.3cm}
\cvitem{\small\textbf{2013}}
	{\textit{The influence of thermal evolution in the magnetic protection of terrestrial planets}, 
	J.I. Zuluaga, \textbf{S. Bustamante}, P.A Cuartas, J.H. Hoyos, ApJ 
	770 23, 2013. \textbf{ADS:} \link{adsabs.harvard.edu/abs/2013ApJ...770...23Z}}
	\vspace{0.3cm}

\section{\textsc{Conference Papers}}

\cvitem{\small\textbf{2014}}
	{\textit{The Local Group in an explicit cosmological context}, J.E. Forero-Romero, Y. 
	Hoffman, \textbf{S. Bustamante}, S. Gottloeber, G. Yepes, XIV Latin American Regional IAU Meeting,
	Revista Mexicana de Astronom�a y Astrof�sica (Serie de Conferencias) Vol. 44, pp. 118-118,  2014.
	\textbf{ADS:} \link{adsabs.harvard.edu/abs/2014RMxAC..44..118F}}
	\vspace{0.3cm}	
\cvitem{\small\textbf{2013}}
	{\textit{Habitability in binary systems}, P.A. Mason, J. Clark,  P.A. Cuartas, J.I. Zuluaga, 
	\textbf{S. Bustamante}, American Astronomical Society, AAS Meeting \# 222, \# 302.05, 2013.
	\textbf{ADS:} \link{adsabs.harvard.edu/abs/2013AAS...22230205M}}
	\vspace{0.3cm}
\cvitem{\small\textbf{2013}}
	{\textit{Habitability in binary systems: the role of UV reduction and magnetic protection}, 
	J. Clark, P.A. Mason, J.I. Zuluaga, P.A. Cuartas, \textbf{S. Bustamante}, American Astronomical 
	Society, AAS Meeting \#222, \#217.03, 2013. \textbf{ADS:} 
	\link{adsabs.harvard.edu/abs/2013AAS...22221703C}}
	\vspace{0.3cm}	

\vspace{0.5cm}
%==================================================================================================

\section{\textsc{Participation in Events}}

\cvitem{\small\textbf{2010}}
	{{{(Poster)}} \textbf{PLYNET, Python Planetary Physics Package}, 
	\textit{II Colombian Congress of Astronomy and Astrophysics}, Bogot�, Colombia, 2010.}
	\vspace{0.3cm}
\cvitem{\small\textbf{2011}}
	{{{(Oral presentation)}} \textbf{Thermal evolution of rocky exoplanets}, 
	\textit{III International Congress of Formation and Modelling in Basic Sciences}, Medell�n, 
	Colombia, 2011.}
	\vspace{0.3cm}
\cvitem{\small\textbf{2012}}
	{{{(Oral presentation)}} \textbf{Numerical modelling of the interior and thermal
	evolution of habitable planets}, \textit{II International Congress of Astrobiology}, Medell�n, 
	Colombia, 2012.}
	\vspace{0.3cm}
\cvitem{\small\textbf{2012}}
	{{{(Oral presentation)}} \textbf{The place of the Local Group in the cosmic web}, 
	\textit{III Colombian Congress of Astronomy and Astrophysics}, Bucaramanga, Colombia, 2012.}
\cvitem{\small\textbf{2014}}
	{{{(Oral presentation)}} \textbf{The fractional anisotropy as a tracer of cosmic voids}, 
	\textit{IV Colombian Congress of Astronomy and Astrophysics}, Pasto, Colombia, 2014.}

\vspace{0.5cm}
%==================================================================================================

\section{\textsc{Schools and Workshops}}

\cvitem{\small\textbf{2014}}
	{\textbf{First Andean School of Astronomy and Astrophysics} (EAAA), Quito, Ecuador, 2014.}

\vspace{0.5cm}
%==================================================================================================

\section{\textsc{Personal Activities}}

\cvitem{\small\textbf{}}
	{General programming. Swimming. Reading (specially science fiction). Playing guitar (amateur 
	level and specially classical guitar). Hiking. }



\vspace{0.5cm}
%==================================================================================================

%\section{\textsc{References}}

%\begin{itemize}

%\item \textbf{Jaime E. Forero-Romero, Ph.D.}\hspace{0.4cm}(\emph{Former Advisor}) \\
%Assistant Professor, Physics Department, Universidad de los Andes, Bogot�, Colombia.\\  
%\href{mailto:je.forero@uniandes.edu.co}{\texttt{je.forero@uniandes.edu.co}}\\

%\item \textbf{Juan Carlos Mu�oz-Cuartas, Ph.D.}\hspace{0.4cm} \\
%Assistant Professor, Physics Institute, Universidad de Antioquia, Medell�n, Colombia.\\  
%\href{mailto:jcmunoz@fisica.udea.edu.co}{\texttt{jcmunoz@fisica.udea.edu.co}}\\  
 
%\item \textbf{Carlos A. Vera-Ciro, Ph.D.}\hspace{0.4cm} \\
%Postdoctoral Researcher, Department of Astronomy, University of Wisconsin-Madison, Madison, 
%Wisconsin, United States.\\  
%\href{mailto:ciro@astro.wisc.edu}{\texttt{ciro@astro.wisc.edu}}\\

%\end{itemize}


\end{document}
