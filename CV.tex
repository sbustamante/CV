\documentclass[10pt,a4paper]{moderncv}
\moderncvtheme[blue]{classic}      
\usepackage[scale=0.8]{geometry}
\usepackage{color}
\definecolor{myblue}{RGB}{55,115,179}
\recomputelengths

%Personal Data
% \firstname{\Large{\textbf{\textit{Sebasti�n}}\vspace*{0.3cm} \\} }
% \familyname{\huge{\textbf{Bustamante Jaramillo}}\\  \color{myblue}\rule{7.5cm}{0.5mm}}
\firstname{\Large{\textbf{Sebasti�n}} }
\familyname{\huge{\textbf{Bustamante Jaramillo}}}
\title{\large{Curriculum Vitae}}
% \address{HITS gGmbH}{Heidelberg, Germany}
% \email{sebastian.bustamante at h-its.org}
% \photo[65pt]{MyPicture.jpg} 
%\quote{\small I am a physicist with a strong interest in astrophysics, specifically in cosmology.
%I have always thought humanity's understanding the large-scale universe through science is like a
%bacteria colony trying to build a complete map of the earth or even more. I am looking forward to 
%cooperate and get involved into high quality research teams in order to contribute my humble grain 
%of sand to this grand and exciting enterprise called science.}
\nopagenumbers{}


%**************************************************************************************************

\begin{document}

\vspace{1.0cm}
\maketitle

%==================================================================================================
\section{\textsc{Contact information}}
%==================================================================================================
\cvitem{\small\textbf{Name:}}
	{Sebasti�n Bustamante Jaramillo}
\cvitem{\small\textbf{Institution:}}
	{Heidelberg Institute for Theoretical Studies (HITS)}
\cvitem{\small\textbf{Place}}
	{Heidelberg, Germany}
\vspace{0.5cm}
\cvitem{\small\textbf{Citizenship:}}
	{Colombian}
\cvitem{\small\textbf{Email:}}
	{sebastian.bustamante@h-its.org}
\cvitem{\small\textbf{Website:}}
	{\url{sbustamante.github.io/Sebastian-Bustamante}}
\cvitem{\small\textbf{Github:}}
	{\url{github.com/sbustamante}}
\cvitem{\small\textbf{Born:}}
	{June 20th 1990, Maceo, Colombia.}
	
	
%==================================================================================================
\section{\textsc{Research employment}}
%==================================================================================================

\cvitem{\small\textbf{2015-present}}
	{ \textbf{PhD candidate at Heidelberg Institute for Theoretical Studies.}}
\cvitem{\small\textbf{}}
	{ 3 years PhD program at Heidelberg Institute for Theoretical Studies (HITS). }
   
  
%==================================================================================================
\section{\textsc{Education and academic degrees}}
%==================================================================================================


\cvitem{\small\textbf{2015-2018$^*$}}
	{\textbf{PhD in Astrophysics at Heidelberg University}}
\cvitem{}
	{$^*$Expected graduation date in the last quarter of 2018.}
\cvitem{}
	{Thesis title: \textit{Modelling supermassive black holes in hydrodynamical simulations of galaxy formation.}}
\cvitem{}
	{Advisor: Prof. Dr. Volker Springel.}

	
	
\cvitem{\small\textbf{2014-2015$^*$	}}
	{\textbf{MSc in Physics at Universidad de Antioquia}}
\cvitem{}
	{$^*$Two terms completed.}
	
	
	
\cvitem{\small\textbf{2007-2012	}}
	{\textbf{BSc in Physics at Universidad de Antioquia}}
\cvitem{}
	{Degree awarded April 4th 2013.}
\cvitem{}
	{Thesis title: \textit{The place of the Milky Way and Andromeda in the cosmic web.}}
\cvitem{}
	{Advisor: Prof. Dr. Jaime E. Forero-Romero.}


\cvitem{\small\textbf{2001-2006}}							
	{\textbf{High School.}} 
\cvitem{\small\textbf{}}
	{Instituci�n Educativa Cisneros, Colombia.} 	


%==================================================================================================
\section{\textsc{Scholarships and prizes}}
%==================================================================================================


\cvitem{\small\textbf{2015}}
	{\textbf{PhD scholarship from the DAAD (German Academic Exchange Service).}}
\cvitem{\small\textbf{}}
	{3 years PhD scholarship for going to HITS. $\sim 45000$ EUR.} 	
\cvitem{\small\textbf{2013}}
	{\textbf{Best BSc physics student of 2012 from Universidad de Antioquia.}}
\cvitem{\small\textbf{}}
	{Exonerated from paying tuition fees of the master program.}
\cvitem{\small\textbf{2012}}
	{\textbf{Best oral presentation at 2nd International Congress of Astrobiology.}}

	
\vspace{2.0cm}
	
%==================================================================================================
\section{\textsc{Computer and technical skills}}
%==================================================================================================

\cvitem{}
	{\begin{itemize}
	\item \textbf{Hydrodynamical simulations:} Running and developing simulations with Gadget 
	and AREPO codes. I have developed a module in AREPO to integrate spin evolution of supermassive 
	black holes in simulations of galaxy formation.
	\item \textbf{Software development:} 
	\begin{itemize}
	\item \textbf{PLYNET:} I have developed a software to calculate the interior structure and thermal 
	evolution of rocky planets. (\url{github.com/facom/Plynet/tree/1.0-release}). 
	\item \textbf{Void Finder:} I also developed a method to find voids in cosmological simulations based 
	on the tidal-tensor and the watershed transform. (\url{github.com/sbustamante/Void-Finder})
	\end{itemize}
	\item \textbf{Programming languages:} Python, C, Bash, Mathematica, TI-Basic.
	\item \textbf{Systems and Software:} Linux, MSWindows, LaTeX, gnuplot.
	\item \textbf{Repositories:} A list of my projects can be found in my \textit{github} page: 
	\url{github.com/sbustamante}
	\end{itemize}}
	

%==================================================================================================
\section{\textsc{Teaching experience}}
%==================================================================================================


\cvitem{\small\textbf{2013}}
	{\textbf{Tutoring at Universidad de Antioquia}:}
\cvitem{}
	{Physics 1, computational lab of Physics 2, computational lab of Physics 3.}
	
\cvitem{\small\textbf{2013-2015}}
	{\textbf{Lecturer at Universidad de Antioquia}: }
\cvitem{}
	{Introduction to physics, lab of Physics 1, introduction to computers, computational methods for Astronomy and Physics (\url{github.com/sbustamante/ComputationalMethods}).}
	
\cvitem{\small\textbf{2017}}
	{\textbf{Tutoring at Heidelberg University}: }
\cvitem{}
	{ Introduction to computational physics. }

	
%==================================================================================================
\section{\textsc{Mentoring}}
%==================================================================================================

\cvitem{}
	{I am the main thesis advisor of Daniel Montenegro. He is a undergrad student of Astronomy at 
	Universidad de Antioquia, in Colombia.}
	
%==================================================================================================
\section{\textsc{Languages}}
%==================================================================================================

\cvitem{\small\textbf{Spanish}}
	{Native speaker.}
\cvitem{\small\textbf{English}}
	{Proficient.}
\cvitem{\small\textbf{German}}
	{Intermediate.}

%==================================================================================================
\section{\textsc{References}}
%==================================================================================================


\cvitem{$\odot$}
	{\textbf{Prof. Volker Springel}}
\cvitem{}
	{\href{mailto:Volker.Springel@h-its.org}{\texttt{Volker.Springel@h-its.org}}}
\cvitem{}
	{Heidelberg Institute for Theoretical Studies, Heidelberg, Germany.}
\cvitem{}
	{Max Planck Institute for Astrophysics, Garching, Germany.}

\vspace{0.5cm}

\cvitem{$\odot$}
	{\textbf{Prof. Jaime E. Forero-Romero}}
\cvitem{}
	{\href{mailto:je.forero@uniandes.edu.co}{\texttt{je.forero@uniandes.edu.co}}}
\cvitem{}
	{Universidad de los Andes, Bogot�, Colombia.}


\vspace{0.5cm}
	
\cvitem{$\odot$}
	{\textbf{Dr. Martin Sparre}}
\cvitem{}
	{\href{mailto:martinsparre@gmail.com }{\texttt{martinsparre@gmail.com }}}
\cvitem{}
	{Potsdam University, Potsdam, Germany.}

\end{document}
