\documentclass[10pt,a4paper]{moderncv}
\moderncvtheme[blue]{classic}      
\usepackage[scale=0.8]{geometry}
\usepackage{color}
\definecolor{myblue}{RGB}{55,115,179}
\recomputelengths

%Personal Data
\firstname{\Large{\textbf{Sebasti�n}\vspace*{0.3cm} \\} }
\familyname{\huge{\textbf{Bustamante Jaramillo}}\\  \color{myblue}\rule{7.5cm}{0.5mm}}
\title{\large{Curriculum Vitae}}
\address{Bello, Antioquia}{Avenida 21 \# 57 AA 65}
\phone{+057 (4) 4820138}
\mobile{+057 3108992409}
\email{macsebas33@gmail.com}
\photo[74pt]{MyPicture.png} 
%\quote{Some nice quote}
\nopagenumbers{}


%**************************************************************************************************

\begin{document}

\vspace{1.0cm}
\maketitle
\begin{minipage}{12.5cm}
\section{\textsc{General Information}}

\cvitem{\small\textbf{Name}}	
	{ Sebasti�n Bustamante Jaramillo.}

\cvitem{\small\textbf{Date of Birth}}	
	{ 20$^{th}$ June, 1990.}

\cvitem{\small\textbf{Place of Birth}}	
	{ Maceo - Antioquia.}

\cvitem{\small\textbf{Nationality}}		
	{Colombian.}

\cvitem{\small\textbf{Identification}}	
	{ c.c. 1128400433 .}

\cvitem{\small\textbf{Address}}			
	{Avenida 21 \# 57 AA 65, Bello - Colombia.}

\cvitem{\small\textbf{Home tel.}}		
	{+057 (4) 4820138 .}

\cvitem{\small\textbf{Mobile}}			
	{+057 3108992409 .}

\cvitem{\small\textbf{Email}}			
	{macsebas33@gmail.com.}

\cvitem{\small\textbf{Inst. Email}}		
	{sbustama@pegasus.udea.edu.co.}
   
\end{minipage}

\vspace{1.0cm}

\subsection{Education}

\cvitem{\small\textbf{2007-2012}}							
	{\textbf{B.Sc. in Physics}, \textit{Institute of Physics, Universidad de Antioquia}, Medell�n, Colombia.} 
	
\cvitem{\small\textbf{Thesis}}
	{``The place of the Milky Way and Andromeda in the cosmic web''.}
	
\cvitem{\small\textbf{Description}}
	{This study is aimed to characterize the local environment of Local Group (LG)-like systems from dark matter 
	simulations of the large-scale universe. Using two different types of simulations, a unconstrained 
	simulation (Bolshoi project) and a set of constrained simulations (CLUES project), it is first constructed
	a LG-like sample based upon observational constrains on the kinematic properties and isolation criteria of 
	the real LG and the results of the CLUES simulations. By using a tensorial scheme based upon the peculiar 
	velocity field of dark matter, the V-web scheme, it is classified the local environment of each simulation. 
	Finally, it has been found that LG-like systems lies preferentially in sheet-like regions, furthermore a 
	significant environmental bias for the total mass and the specific energy. No correlations have been found 
	for the specific angular momentum and other studied properties.}

\vspace{0.5cm}
\subsection{Fields of Interest}

\cvitem{}{Tropical Meteorology, moisture processes, circulation patterns.}
\vspace{0.8cm}
\section{Occupational and academic Experience}

\cvitem{2013-Present}{\underline{Teacher,Geostrophic fluid dynamics}, Oceanography, Universidad de Antioquia.}
\cvitem{2013-Present}{\underline{Teacher, Introductory physics}, Faculty of Engineering, Universidad de Antioquia.}
\cvitem{2012}{\underline{Teacher, Introductory physics for highschool Students}, Universidad de Antioquia.}
\cvitem{2009}{\underline{Museum guide of Galileo interactive physics room}, Museo Universitario de la  Universidad de Antioquia.}
%\cvitem{2011-2012}{\underline{Teacher, Solid State Physics}, Faculty of Engineering, Universidad de Antioquia.}
%\cvitem{2007-2009}{\underline{Teaching assistant for introductory physics courses} Faculty of Engineering, Universidad de Antioquia.}
\vspace{1.0cm}

\section{Research and Professional Activities}
 
%  My currently research topic is in  magnetic properties of  manganite thin films doped with Fe and Cr with emphasis in magnetic transitions Ferro-Para,
% using critical phenomena theory. Additionally i have been studied thermoelectric materials like $CoSb_3$: synthesis, structural and transport 
% properties. 


\cvitem{2012-2013}{\textbf{Moisture flux over Colombia in the period 1979-2009 using data from the ERA-Interim reanalysis}, Main researcher,
Committee for Research Development (CODI), Universidad de Antioquia, Universidad Nacional de Colombia, sede Bogot�.}

\cvitem{2010-2011}{\textbf{Analysis of extreme climate events and Circulation patterns over Colombia using the regional climate model REMO}, Training
student, Universidad de Antioquia.}


 
\vspace{0.5cm}
\subsection{Conferences, schools and workshops attended}
\vspace{0.3cm}


\cvitem{2012}{\textbf{Regional WCRP/SPARC Workshop with focus on the Southern Hemisphere and South America.},
Universidad de Buenos Aires,November 26-27, Buenos Aires, Argentina.}
\cvitem{Poster:}{Moisture Flux across Northerm of South America using ERA Interim data.}

\cvitem{2012}{\textbf{XI Meeting of teaching in science and mathematics.},
 Universidad de Antioquia, Universidad EAFIT, April 19-20, Medell�n, Colombia.}

 \cvitem{2011}{\textbf{XI Colombian Congress of Meteorology.},
 Universidad Nacional de Colombia, sede Bogot�, March 23-25, Bogot�, Colombia.}
 
 \cvitem{2008}{\textbf{I Colombian Congress of Astronomy and Astrophysics.},
 Universidad de Antioquia, Universidad EAFIT, August 19-20, Medell�n, Colombia.}

\subsection{Miscellaneous}
\vspace{0.3cm}
\cvitem{Computer Skills}{ C, Linux, Mathematica, CDO,GrADS .}
\cvitem{Languages}{ Spanish (native), English. }

% \section{ENA}
% 
% 
% It would be a great opportunity to take part in the school 
% ``Exploring Nanomagnetism and its Applications'' (ENA), it will help me  to  improve my knowledge in general magnetism ideas to be applied to my
%  research  and futures projects. The quality and recognition of speakers makes this school an opportunity to establish academic contacts with University of Antioquia , specially with
% Solid State Physics Research Group.
\vspace{1.0cm}
\section{References}

\begin{itemize}


\item \textbf{Jaime E. Forero-Romero }\hspace{0.4cm}(\emph{Advisor}) \\
 Professor and Researcher, Foundations and Teaching Group of Physics and Dynamical Systems,  Atomic and Molecular Physics Group.\\
  Instituto de F�sica, \underline{Universidad de Antioquia}, Medell�n, Colombia.\\
  Phone: 57-4-219 64 39.\\  
  \href{mailto:banghelo@fisica.udea.edu.co}{\texttt{banghelo@fisica.udea.edu.co}}\\
  
\item \textbf{Isabel Cristina Hoyos Rinc�n }\hspace{0.4cm}(\emph{Advisor}) \\
  Researcher, Foundations and Teaching Group of Physics and Dynamical Systems.\\
  Instituto de F�sica, \underline{Universidad de Antioquia}, Medell�n, Colombia.\\
%  Phone: 57-4-219 65 75, 57-4-219 56 33.\\  
  \href{mailto:hoyos.isabel@gmail.com}{\texttt{hoyos.isabel@gmail.com}}\\
  %\href{mailto:oarnache@gmail.com}{\texttt{oarnache@gmail.com}}\\
 

\end{itemize}


\end{document}
