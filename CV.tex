\documentclass[10pt,a4paper]{moderncv}
\moderncvtheme[blue]{classic}      
\usepackage[scale=0.8]{geometry}
\usepackage{color}
\definecolor{myblue}{RGB}{55,115,179}
\recomputelengths

%Personal Data
\firstname{\Large{\textbf{\textit{Sebasti�n}}\vspace*{0.3cm} \\} }
\familyname{\huge{\textbf{Bustamante Jaramillo}}\\  \color{myblue}\rule{7.5cm}{0.5mm}}
\title{\large{Curriculum Vitae}}
\address{Bello, Colombia}{Avenida 21 \# 57 AA 65}
\phone{+057 (4) 4820138}
\mobile{+057 3108992409}
\email{macsebas33@gmail.com}
\photo[74pt]{MyPicture.png} 
\quote{\small I am a physicist with a strong interest in astrophysics, specifically in cosmology.
I have always thought humanity's understanding the large-scale universe through science is like a
bacteria colony trying to build a complete map of the earth or even more. I am looking forward to 
cooperate and get involve with high quality research teams in order to contribute my humble grain 
of sand to this grand and exciting enterprise called science. If I were asked to introduce myself
in four words, they would be \textit{astrophysics-programming-swimming-guitar}.}
\nopagenumbers{}


%**************************************************************************************************

\begin{document}

\vspace{1.0cm}
\maketitle
%==================================================================================================

\section{\textsc{General Information}}

\cvitem{\small\textbf{Name}}	
	{ Sebasti�n Bustamante Jaramillo.}
\cvitem{\small\textbf{Date of Birth}}	
	{ 20$^{th}$ June, 1990.}
\cvitem{\small\textbf{Place of Birth}}	
	{ Maceo - Colombia.}
\cvitem{\small\textbf{Nationality}}		
	{ Colombian.}
\cvitem{\small\textbf{Marital Status}}	
	{ Single.}
\cvitem{\small\textbf{Identification}}	
	{ c.c. 1128400433 .}
\cvitem{\small\textbf{Address}}			
	{Avenida 21 \# 57 AA 65, Bello - Colombia.}
\cvitem{\small\textbf{Home tel.}}		
	{+057 (4) 4820138 .}
\cvitem{\small\textbf{Mobile}}			
	{+057 3108992409 .}
\cvitem{\small\textbf{e-mail}}			
	{macsebas33@gmail.com.}
\cvitem{\small\textbf{Inst. e-mail}}		
	{sbustama@pegasus.udea.edu.co.}
   

\vspace{0.5cm}
%==================================================================================================

\section{\textsc{Education}}

\cvitem{\small\textbf{2001-2006}}							
	{\textbf{High School Diploma}, \textit{Instituci�n Educativa Cisneros}, Cisneros, Colombia.} 	

\cvitem{\small\textbf{2007-2012}}							
	{\textbf{B.Sc. in Physics}, \textit{Institute of Physics, Universidad de Antioquia}, Medell�n, 
	Colombia.} 	
\cvitem{\small\textbf{Thesis}}
	{``The place of the Milky Way and Andromeda in the cosmic web''.}
\cvitem{\small\textbf{Description}}
	{This study is aimed to characterize the local environment of Local Group (LG)-like systems from 
	dark matter simulations of the large-scale universe. Using two different types of simulations, an 
	unconstrained simulation (Bolshoi project) and a set of constrained simulations (CLUES project), 
	it is first constructed a LG-like sample based upon observational constrains on the kinematic 
	properties and isolation criteria of the real LG, along with the results of the CLUES simulations. 
	By using a tensorial scheme based upon the peculiar velocity field of the dark matter, the V-web 
	scheme, it is classified the local environment of systems in each simulation. Finally, it has 
	been found that LG-like systems lies preferentially in sheet-like regions, furthermore a 
	significant	environmental bias for the total mass and the specific energy. No correlations have 
	been found for the specific angular momentum and other studied properties.}
\cvitem{\small\textbf{Advisor}}
	{Jaime E. Forero-Romero, Ph.D.}


\vspace{0.5cm}
%--------------------------------------------------------------------------------------------------
\subsection{\textsc{Additional Education}}
\cvitem{\small\textbf{2008}}							
	{\textbf{Extension Course of Planetary Sciences}, \textit{Faculty of Exact and Natural Sciences, 
	Universidad de Antioquia}, Medell�n, Colombia.}


\vspace{0.5cm}
%==================================================================================================

\section{\textsc{Fields of Interest}}

\cvitem{}
	{Cosmology. Large-scale structure formation. Galaxy astrophysics. Planetary interior. Numerical 
	simulations. Computational astrophysics. General astrophysics and physics. Programming.}


\vspace{0.5cm}
%==================================================================================================

\section{\textsc{Languages}}

\cvitem{\small\textbf{Spanish}}
	{Native speaker.}
\cvitem{\small\textbf{English}}
	{Good.}


\vspace{0.5cm}
%==================================================================================================

\section{\textsc{Computer Skills}}

\cvitem{\small\textbf{Systems}}
	{Linux, MSWindows.}
\cvitem{\small\textbf{Development}}
	{C/C++, Python, Basic, TI Basic, shell scripts.}
\cvitem{\small\textbf{Software}}
	{Mathematica, LaTeX, gnuplot, Gadget.}
\cvitem{\small\textbf{Tools}}
	{N-body simulations, SPH, MonteCarlo, Finite Differences, Numerical integrators, Audio 
	processing.}
\cvitem{\small\textbf{Repositories}}
	{A list of my projects can be found in my \textit{github} page: 
	\url{https://github.com/sbustamante}.}


\vspace{0.5cm}
%==================================================================================================

\section{\textsc{Honours, Awards, and Accomplishments}}

\cvitem{\small\textbf{2006}}					
	{\textbf{The best graduate of the promotion}, \textit{Instituci�n Educativa Cisneros}, 
	Cisneros, Colombia.}
\cvitem{\small\textbf{2007/II}}					
	{\textbf{Honour Roll}, \textit{Universidad de Antioquia}, Medell�n, Colombia.}
\cvitem{\small\textbf{2009/I}}					
	{\textbf{Honour Roll}, \textit{Universidad de Antioquia}, Medell�n, Colombia.}
\cvitem{\small\textbf{2009/II}}					
	{\textbf{Honour Roll}, \textit{Universidad de Antioquia}, Medell�n, Colombia.}
\cvitem{\small\textbf{2012}}
	{\textbf{First Best Oral Presentation}, \textit{II International Congress of Astrobiology}, 
	Medell�n, Colombia.}

\vspace{0.5cm}
%==================================================================================================

\section{\textsc{Teaching}}

\cvitem{\small\textbf{Assistant Instructor}}
	{\textbf{Physics 1 (Newtonian Mechanics), 2013/I}, \textit{Faculty of Exact and Natural Sciences, 
	Universidad de Antioquia}, Medell�n, Colombia.}	
\cvitem{\small\textbf{Assistant Instructor}}
	{\textbf{Computational Complement of Physics 2 (Electricity and Magnetism), 2013/I}, 
	\textit{Faculty of Exact and Natural Sciences, Universidad de Antioquia}, Medell�n, Colombia.}
\cvitem{\small\textbf{Assistant Instructor}}
	{\textbf{Computational Complement of Physics 3 (Oscillations and Waves), 2013/I}, 
	\textit{Faculty of Exact and Natural Sciences, Universidad de Antioquia}, Medell�n, Colombia.}
\cvitem{\small\textbf{Adjunct Professor}}
	{\textbf{Introductory Physics, 2013/II}, \textit{Faculty of Engineering, Universidad de 
	Antioquia}, Medell�n, Colombia.}
\cvitem{\small\textbf{Adjunct Professor}}
	{\textbf{Laboratory of Physics 1 (Newtonian Mechanics), 2013/II}, \textit{Faculty of Engineering, 
	Universidad de Antioquia}, Medell�n, Colombia.}


\vspace{0.5cm}
%==================================================================================================

\section{\textsc{Research Experience}}
 
\cvitem{\small\textbf{2010-2011}}	
	{\textbf{Young Investigator Programme}, Fundamental of quantum mechanics, \textit{Group of Atomic 
	and Molecular Physics (GFAM), Committee for Research Development (CODI), Universidad de Antioquia}, 
	Medell�n, Colombia.}
\cvitem{\small\textbf{2011-2012}}	
	{\textbf{Young Investigator Programme}, Thermal evolution of rocky exoplanets, \textit{Group of
	Computational Physics and Astrophysics (FACom), Committee for Research Development (CODI), 
	Universidad de Antioquia}, Medell�n, Colombia.}
\cvitem{\small\textbf{August, 2012}}	
	{\textbf{Research Internship}, \textit{Physics Department, Universidad de los Andes}, Bogot�, 
	Colombia.}
 
 
\vspace{0.5cm}
%==================================================================================================

\section{\textsc{Publications}}

\cvitem{\small\textbf{1}}
	{\textit{The Influence of Thermal Evolution in the Magnetic Protection of Terrestrial Planets}, 
	J.I. Zuluaga, \textbf{S. Bustamante}, P.A Cuartas, J.H. Hoyos, ApJ 770 23, 2013.}
\cvitem{\small\textbf{2}}
	{\textit{The kinematics of the Local Group in a cosmological context}, J.E. Forero-Romero, Y. 
	Hoffman, \textbf{S. Bustamante}, S. Gottloeber, G. Yepes, ApJL 767 L5, 2013.}

\vspace{0.3cm}


\vspace{1.0cm}
\section{References}

\begin{itemize}


\item \textbf{Jaime E. Forero-Romero }\hspace{0.4cm}(\emph{Advisor}) \\
 Professor and Researcher, Foundations and Teaching Group of Physics and Dynamical Systems,  Atomic and Molecular Physics Group.\\
  Instituto de F�sica, \underline{Universidad de Antioquia}, Medell�n, Colombia.\\
  Phone: 57-4-219 64 39.\\  
  \href{mailto:banghelo@fisica.udea.edu.co}{\texttt{banghelo@fisica.udea.edu.co}}\\
  
 

\end{itemize}


\end{document}
